\documentclass[12pt,a4paper,notitlepage]{article}
\usepackage[utf8]{inputenc}
\usepackage[english]{babel}
\usepackage{amsmath}
\usepackage{amsfonts}
\usepackage{amssymb}
\usepackage{graphicx}
\title{Reputation d'objets et fiabilités des juges}
\author{Malian De Ron \and Quentin Laurent}
\begin{document}

\maketitle

\subsection*{Introduction}

\subsection*{Objectifs}

Nous avons l'intention de fonctionner en 3 étapes distinctes afin de pouvoir faire évoluer le projet tout en ayant des exemples à chaque étape.\\

Si possible nous nous baserons sur des données réelles, mentionnées ci-dessous, et si ce n'est pas possible nous générerons aléatoirement des données réalistes.

\subsubsection*{Comparaison du modèle de base et du modèle à moyenne tronquée}

Dans un premier temps nous aller effectuer une simple comparaison entre le modèle qui calcule de manière itérative la réputation de m objets cotés par n juges et le modèle empirique qui supprime la meilleur et la plus mauvaise cote de chaque objets et qui effectue la moyenne sur les cotes restantes.\\

Nous espérons observer soit une différence remarquable entre les deux modèles soit que les deux modèles se comportent de manière semblable pour de petits ensembles de données telles que les côtes attribuées aux jeux olympiques ou des cotes attribuées à des élèves.

\subsubsection*{Modèle multivarié d'un juge qui cote m objets à n composantes}

\subsubsection*{Modèle multivarié d'un juge qui cote m objets à n composantes en fonction de k points de vues}

\subsection*{Pour aller plus loins}

Introduction de spam

\subsection*{Conclusion}

\end{document}
