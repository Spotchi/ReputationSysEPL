\begin{titlepage}
\centering
\BgThispage
\newgeometry{left=1cm,right=1cm, top=1cm, bottom=1cm}
\vspace*{3.5cm}
\noindent
\begin{spacing}{2.5}\textcolor{white}{\bigsf  Multi-characteristics reputation system using iterative filtering}\par\end{spacing}
\vspace*{2.6cm}\par
\noindent
%
\begin{minipage}{0.30\linewidth}
    \begin{flushright}
        \textbf{De Ron Malian} \small (40220800)
        \\\textbf{Laurent Quentin}  \small (48341100)\\
        \textbf{Promotor:} Paul Van Dooren
    \end{flushright}
\end{minipage} \hspace{15pt}
%
\begin{minipage}{0.02\linewidth}
    \rule{1pt}{100pt}
\end{minipage} \hspace{-10pt}
%
\begin{minipage}{0.6\linewidth}
    \begin{abstract}
    \begin{normalsize}
        In this work we present a reputation system in which $N$ judges rate. This system assigns a reputation to each one of a set of $M$ objects in each one of a set of $K$ characteristics. It also computes     a weight for each judge. Those reputations and weights are obtained using an iterative algorithm. In this work we derive the linear convergence of this iterative process towards a unique point. We also analyse the influence of the modification of ratings on reputations. In order to validate the method we compare it with outlier detection and classic mean, and introduce fake partial judges. The method indeed diminishes their influence, although some fake judges with the most extreme ratings can prevent the detection of other more subtle cheaters.
    \end{normalsize}
    \end{abstract}
\end{minipage}
\vspace{1.5cm} \begin{center}
        \today
        \end{center}
\vspace*{1cm}
\centering
\begin{minipage}{0.2\linewidth}
	\includegraphics[scale=0.2]{images/logo_ucl.jpg}
\end{minipage} \hspace{-10pt}
\begin{minipage}{0.2\linewidth}
\centering
    \includegraphics[scale=0.2]{images/logo_epl.jpg}
\end{minipage}
\end{titlepage}
\restoregeometry
