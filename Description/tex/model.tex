In this part we will build a rating system for $N$ judges evaluating $M$ objects, each having a set of $K$ different characteristics. 
The judges give the objects some set of ratings $x_{ijk}$.
At the end of the iterative process, each rating $i$ will be given a weight $w_{ijk}$ and each object $j$ will be given a reputation $r_{jk}$.


\subsection*{Description of general filtering methods}
As defined in other works, an iterative filtering(IF) system is composed of two basic functions \cite{Cristo1} : 
\begin{itemize}
\item The reputation function : $F(w,X)=r$
\item The filtering function : $G(r,X)=w$
\end{itemize}
The two of them define an iterative filtering system.\\
In quadratic IF systems, the reputation function is naturally given by the weighted average of the votes.
$$F_{ik}(w,X) = \frac{\sum_{i}x_{ijk}w_{ijk}}{\sum_i w_{ijk}}$$

The filtering method used in the aforementioned paper adapted to a multi-variate system is $G_{ik}(w,X) = \log \prod_j f(x_{ijk}|r,C)$

\subsection*{Filtering algorithm specification}
Let us remember the guidelines we had for the choice of our filtering algorithm. It should be able to handle several cases :
\begin{itemize}
\item A judge rating one object much better than the others in a caracteristic should be given less influence
\item A judge rating with a mean rating above or below the average of the other judges should have his points adjusted or his influence in the final rating diminished.
\item The variance of the ratings for a given judge and caracteristic should also be included in the equation
\end{itemize}

\subsection*{Proposed method}
At first we could be tempted to simply adapt the iterative filtering to each characteristic independently. This would of course lead to the basic scheme described in \cite{Cristo1} for each characteristic.\\
In order to be more complete, we need to include the dependence of one characteristic according to another.
The model we propose uses the correlation between two characteristics to address this issue.
Let $C_i$ be the covariance matrix between characteristics for the judge $i$.
We define the "partiality" function as follows :
$$\gamma_{i}(X,r,\Delta,C) = \log (\prod_j \sqrt{\frac{1}{(2\pi)^{MK}\det C}} \exp^{ (X_{ij}-r_j-\Delta_i)^TC^{-1} (X_{ij}-r_j-\Delta_i)})$$
Where $X_{ij}$ is the column-vector of all the ratings given by the judge $i$ for object $j$, $r_j$ is the column-vector of the reputation of object $j$ in all characteristics and $\Delta_i$ is the column-vector of bias for $i$.\\
The "bias" function is denoted as $\beta(\Delta)$. It is used to penalize too low or high means. It is still to be proposed.\\
We could also influence on the weights for each rating by introducing some tuning parameter $\psi_{jk}$.
Thus the final filtering function would be given by 
$$G_{ijk}(r,X) = \gamma_{i}(X,r,\Delta,C) + \beta_i(\Delta)+\psi_{jk}$$

As we can see, this iteration scheme should satisfy the specifications : 
\begin{itemize}
\item A judge that has an inconsistent rating to an object will have a lower weight. If the covariance between two characteristics is strictly positive, the reputation in two different characteristics tend to increase together.
\item The bias function handles the average of some judges that could be deemed as inappropriate.
\item Different variances lead to different penalizations in the scheme (a higher variance for a judge and characteristic means we are less severe with this judge).
\end{itemize}

%\begin{table}[h]
%\centering
%\begin{tabular}{|c|c|c|}
%\hline 
%Name & Indexing  & Description \\ 
%\hline
%$N$ &  & The number of judges \\ 
%
%$M$ &  & The number of objects \\ 
%$K$ &  & The number of characteristics \\ 
%$x_{ijk}$ & \parbox[t]{3cm}{$\forall i \in \left\lbrace 1,...,N\right\rbrace$,%\newline
%$j \in \left\lbrace 1,...,M\right\rbrace$,\newline
%$k \in \left\lbrace 1,...,K\right\rbrace$ } &  \parbox[t]{5cm}{The rating given by %judge $i$ to characteristic $k$ of object $j$} \\ 
%$R_{jk}$ & 
%\parbox[t]{3cm}{$\forall j \in \left\lbrace 1,...,M\right\rbrace$,\newline
%$k \in \left\lbrace 1,...,K\right\rbrace$ } & \parbox[t]{5cm}{The intrinsic value of %characteristic $k$ of object $j$}\\ 
%$\Delta_{ijk}$ & 
%\parbox[t]{3cm}{$\forall i \in \left\lbrace 1,...,N\right\rbrace$,\newline
%$j \in \left\lbrace 1,...,M\right\rbrace$,\newline
%$k \in \left\lbrace 1,...,K\right\rbrace$ }
%& \parbox[t]{5cm}{Bias of rating for the rating given by judge $i$ to characteristic %$k$ of object $j$} \\
%$\mu_{ijk}$ & 
%\parbox[t]{3cm}{$\forall i \in \left\lbrace 1,...,N\right\rbrace$,\newline
%$j \in \left\lbrace 1,...,M\right\rbrace$,\newline
%$k \in \left\lbrace 1,...,K\right\rbrace$ } 
%& \parbox[t]{5cm}{Mean of rating for the rating given by judge $i$ to characteristic %$k$ of object $j$($\mu_{ijk} = Q_{jk} + \Delta_{ijk}$)}\\
%$\sigma_{i_1k_1i_2k_2}^2$ &
%\parbox[t]{3cm}{$\forall i_1,i_2 \in \left\lbrace 1,...,M\right\rbrace$,\newline
%$k_1,k_2 \in \left\lbrace 1,...,K\right\rbrace$ }
% & \parbox[t]{5cm}{Variance of rating for the judge $i$ for characteristic $k$}\\
%$w_{ijk}$ & & \parbox[t]{5cm}{Weight of the rating $ijk$}\\
%\hline 

%\end{tabular} 
%\caption{Notations}\label{table:notation}
%\end{table}