\documentclass[12pt,a4paper,notitlepage]{article}
\usepackage[utf8]{inputenc}
\usepackage[english]{babel}
\usepackage{amsmath}
\usepackage{amsfonts}
\usepackage{amssymb}
\usepackage{graphicx}
\title{Description of the project : Reputation of objects and reliability of judges}
\author{Malian De Ron \and Quentin Laurent}
\begin{document}

\maketitle
\subsection*{Comparison between filtering methods}
How does the presented method compare with , e.g. the outlier method (suppress the lowest and highest ratings).
\subsection*{Robustness of the method}
Analyze a range of behaviours for spammers and cheaters.\\
Currently, C. de Kerckhove only analysed the results for spammers using a 1 grade for an item and a zero grade for the others. We are going to check the robustness of the method for more complex strategies. Is there an optimal behaviour for a cheater in order to increase the reputation of a given set of objects?
\subsection*{Multi-variate version of the reputation system}
Objects will be rated on several aspects, building a more complex object profile. The vector of characteristics of the object can be sparse or full.
Adapt the results of convergence, uniqueness of the solution, and so on ...
\subsection*{Application : Grades of the EPL}
Each teacher will get a reliability measure at the end of the project.
\subsubsection*{Normalizing the data}
Every teacher has a different way of grading. Some tend to give lower marks, some higher, some tend to give everyone similar marks while others have a great variance in giving their marks. We need a way to take in account the mark distribution of the teachers.\\
My proposition : marks according to quantiles. No, bad idea, doesn't take in account the possible mark gaps. Should we find a way to put every rater at the same mean and variance?
\subsubsection*{A profile of the students}
In general, a good student will be expected to perform well in most of his courses. If some chemistry teacher gives the mentioned student a lower mark, it will be far off the mean of the student and lower the reputation of the teacher.\\
But sometimes even good students deserve bad marks, they simply are bad in a given field, here chemistry. Hence it is necessary to build the student's profile, where characteristics are his strength in different broad fields. It is now expected that the marks of the student in chemistry will be lower than his average, and the reputation of chemistry teachers of the EPL won't collapse.

\subsection*{Time-varying reliability}
Not sure yet
\end{document}