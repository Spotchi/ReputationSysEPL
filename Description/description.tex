\documentclass[12pt,a4paper,notitlepage]{article}
\usepackage[utf8]{inputenc}
\usepackage[english]{babel}
\usepackage{amsmath}
\usepackage{amsfonts}
\usepackage{amssymb}
\usepackage{graphicx}
\title{Description of the project : Reputation of objects and reliability of judges}
\author{Malian De Ron \and Quentin Laurent}
\begin{document}

\maketitle
\subsection*{Comparison between filtering methods}
How does the presented method compare with , e.g. the outlier method (suppress the lowest and highest ratings).
\subsection*{Robustness of the method}
Analyze a range of behaviours for spammers and cheaters.\\
Currently, C. de Kerckhove only analysed the results for spammers using a 1 grade for an item and a zero grade for the others. We are going to check the robustness of the method for more complex strategies. Is there an optimal behaviour for a cheater in order to increase the reputation of a given set of objects?
\subsection*{Multi-variate version of the reputation system}
Objects will be rated on several aspects, building a more complex object profile. The vector of characteristics of the object can be sparse or full.
Adapt the results of convergence, uniqueness of the solution, and so on ...
\subsection*{Application : Grades of the EPL}
Each teacher will get a reliability measure at the end of the project.
\subsubsection*{Normalizing the data}
Every teacher has a different way of grading. Some tend to give lower marks, some higher, some tend to give everyone similar marks while others have a great variance in giving their marks. We need a way to take in account the mark distribution of the teachers.\\
My proposition : marks according to quantiles. No, bad idea, doesn't take in account the possible mark gaps. Should we find a way to put every rater at the same mean and variance?
\subsubsection*{A profile of the students}
In general, a good student will be expected to perform well in most of his courses. If some chemistry teacher gives the mentioned student a lower mark, it will be far off the mean of the student and lower the reputation of the teacher.\\
But sometimes even good students deserve bad marks, they simply are bad in a given field, here chemistry. Hence it is necessary to build the student's profile, where characteristics are his strength in different broad fields. It is now expected that the marks of the student in chemistry will be lower than his average, and the reputation of chemistry teachers of the EPL won't collapse.

\subsection*{Time-varying reliability}
Not sure yet

\section*{Model and notations}
In this part we will build a rating system for $N$ judges evaluating $M$ objects, each having a set of $K$ different characteristics. 
The judges give the objects some set of ratings $x_{ijk}$.
At the end of the iterative process, each rating $i$ will be given a weight $w_{ijk}$ and each object $j$ will be given a reputation $r_{jk}$.


\subsection*{Description of general filtering methods}
As defined in other works, an iterative filtering(IF) system is composed of two basic functions \cite{Cristo1} : 
\begin{itemize}
\item The reputation function : $F(w,X)=r$
\item The filtering function : $G(r,X)=w$
\end{itemize}
The two of them define an iterative filtering system.\\
In quadratic IF systems, the reputation function is naturally given by the weighted average of the votes.
$$F_{jk}(w,X) = \frac{\sum_{i}x_{ijk}w_{ijk}}{\sum_i w_{ijk}}$$

The filtering method used in the aforementioned paper adapted to a multi-variate system is $G_{ik}(w,X) = \log \prod_j f(x_{ijk}|r,C)$

\subsection*{Filtering algorithm specification}
Let us remember the guidelines we had for the choice of our filtering algorithm. It should be able to handle several cases :
\begin{itemize}
\item A judge rating one object much better than the others in a caracteristic should be given less influence
\item A judge rating with a mean rating above or below the average of the other judges should have his points adjusted or his influence in the final rating diminished.
\item The variance of the ratings for a given judge and caracteristic should also be included in the equation
\end{itemize}

\subsection*{Proposed method}
At first we could be tempted to simply adapt the iterative filtering to each characteristic independently. This would of course lead to the basic scheme described in \cite{Cristo1} for each characteristic.\\
In order to be more complete, we need to include the dependence of one characteristic according to another.
The model we propose uses the correlation between two characteristics to address this issue.
Let $C_i$ be the covariance matrix between characteristics for the judge $i$.
We define the "partiality" function as follows :
$$\gamma_{i}(X,r,\Delta,C) = \log (\prod_j \sqrt{\frac{1}{(2\pi)^{K}\det C}} \exp^{- (X_{ij}-r_j-\Delta_i)^TC^{-1} (X_{ij}-r_j-\Delta_i)/2})$$
Where $X_{ij}$ is the column-vector of all the ratings given by the judge $i$ for object $j$, $r_j$ is the column-vector of the reputation of object $j$ in all characteristics and $\Delta_i$ is the column-vector of bias for $i$.\\
The "bias" function is denoted as $\beta(\Delta)$. It is used to penalize too low or high means. It is still to be proposed.\\
We could also influence on the weights for each rating by introducing some tuning parameter $\psi_{jk}$.
Thus the final filtering function would be given by 
$$G_{ijk}(r,X) = \gamma_{i}(X,r,\Delta,C) + \beta_i(\Delta)+\psi_{jk}$$

As we can see, this iteration scheme should satisfy the specifications : 
\begin{itemize}
\item A judge that has an inconsistent rating to an object will have a lower weight. If the covariance between two characteristics is strictly positive, the reputation in two different characteristics tend to increase together.
\item The bias function handles the average of some judges that could be deemed as inappropriate.
\item Different variances lead to different penalizations in the scheme (a higher variance for a judge and characteristic means we are less severe with this judge).
\end{itemize}

\subsection*{Corrected scheme}
$$F_{jk}(w,X) = \frac{\sum_{i}x_{ijk}w_{i}}{\sum_i w_{i}}$$
$$G_{i}(X,r,C) = \log (\prod_j \sqrt{\frac{1}{(2\pi)^{K}\det C}} \exp^{- (X_{ij}-r_j-\Delta_i)^TC^{-1} (X_{ij}-r_j-\Delta_i)/2})$$
which is equivalent to
$$G_{i}(X,r,C) = 1 - \frac{\sum_i (X_{i,j,:}-r{j,:})^TC^{-1}(X_{i,j,:}-r{j,:})}{N/2(-\log(2\pi )^K \det C)}$$
$$G_{i}(X,r,C) = 1 -k superd_i$$
with $k= \frac{2}{(-\log(2\pi )^K \det C)}$ and $superd_i =  \frac{1}{N}\sum_{j} (X_{i,j,:}-r_{j,:})^T C^{-1} (X_{i,j,:}-r_{j,:})$


\subsection*{Energy function for the method}

\begin{tabular}{|c|c|}
\hline 
Tensor & size\\
\hline
$X$ & $N\times M \times K$\\
\hline
$R$ & $M\times K$\\
\hline
$w$ & $N\times 1$\\
\hline
$C$ & $K\times K$\\
\hline
\end{tabular}

We can see that a fixed point $(x,w)$ of the proposed method is a solution of the following equation :
$$ r_{:,k}^{t+1} (\mathbf{1}^Tw^{star}) = X_{:,:,k}w^{\star} \: \forall k$$
and 
$$ w^{\star} = G(r^{\star})$$

Hence we get from this that
$$(r_{:,k}^{\star} \mathbf{1}^T - X_{:,:,k})\cdot G(r^{\star}) = 0 \: \forall k\in \{1,..,K\}$$


$$g(u) = 1 -ku$$

\begin{eqnarray*}
E(r) & = & \sum_{i=1}^n \int_0^{superd_i(r_j)}g(u) du\\
\frac{\partial E}{\partial r_{jk}} & = & \sum_{i}\frac{\partial E}{\partial superd_i} \cdot \frac{\partial superd_i}{\partial r_{jk}}
\end{eqnarray*}

\begin{eqnarray*}
\frac{\partial superd_i}{\partial r_{jk}} & = & \left[-2 C^{-1}(X_{i,j,:}-r_{j,:})\right]_{k} \\
\frac{\partial E}{\partial superd_i} & = & g(superd_i)\\
\frac{\partial E}{\partial r_{jk}} & = & \sum_{i}g(superd_i) \cdot \left[-2 C^{-1}(X_{i,j,:}-r_{j,:})\right]_{k} 
\end{eqnarray*}

\paragraph{Case $C = I$}

When $C$ is the identity matrix, we simply get that the condition $\frac{\partial E}{\partial r^{\star}_{jk}}=0 \: \forall j,k$ is equivalent to the fact that $(r^{\star},G(r^{\star}))$ is a stationary point of the iteration.





\section{Uniqueness of the stationnary point}

We can prove that the stationnary point corresponding to our problem is unique under some assumptions on $k$.
We need $$k\in \mathcal{K} = \{k\in \mathcal{R}_{\geq 0} | 1 - k \frac{1}{n}\begin{pmatrix} superd_1 \\ superd_2 \\ \vdots \\ superd_n \end{pmatrix} >0 \: \forall r \in \mathcal{H} \}$$
where $\mathcal{H}$ is an hypercube.

The function $E(r)$ can also be written as 
$$ E(r) = \sum_{i=1}^N superd_i - k \frac{superd_i^2}{2} + c$$
If we take $c = \frac{2N}{k}$
Indeed, we then have $$ \sum_{i=1}^N superd_i - k \frac{superd_i^2}{2} + \frac{2N}{k} = -\frac{1}{2k} (1 - 2ksuperd_i + superd_i^2)$$
\\

There is a lemma in \cite{Cristo1} that goes as follows
\begin{lemma}
Let the function $E(r) : \mathcal{R}^n \rightarrow \mathcal{R} : E(r) = z $ be a fourth-order polynomial and let $\mathcal{H}$ be some hypercube in $\mathcal{R}^n$. If 
$$\lim_{||r||\rightarrow \infty} E(r) = - \infty $$
and the steepest descent direction on the boundary of $\mathcal{H}$ points strictly inside $\mathcal{H}$, then $E$ has a unique stationary point in $\mathcal{H}$ which is a minimum.
\end{lemma}
From which follows the following theorem
\begin{theorem}
If $k \in \mathcal{K}$, the system has a unique fixed point $r^{\star}$.
\begin{proof}
The proof is similar to the uni-variable case. We note that it is more likely in our multi-variate case to have some unique rating for some characteristic for an object. However, the given rating will determine the final reputation since the change in the weight of the judge won't change the reputation. In other cases where the characteristic of the object is rated by two judges, it will go to the interior and we can see that $E(r)$ has a unique stationary point in $\mathcal{H}$ which is a minimum and that it is the unique fixed point of the system.
\end{proof}
\end{theorem}




%\begin{table}[h]
%\centering
%\begin{tabular}{|c|c|c|}
%\hline 
%Name & Indexing  & Description \\ 
%\hline
%$N$ &  & The number of judges \\ 
%
%$M$ &  & The number of objects \\ 
%$K$ &  & The number of characteristics \\ 
%$x_{ijk}$ & \parbox[t]{3cm}{$\forall i \in \left\lbrace 1,...,N\right\rbrace$,%\newline
%$j \in \left\lbrace 1,...,M\right\rbrace$,\newline
%$k \in \left\lbrace 1,...,K\right\rbrace$ } &  \parbox[t]{5cm}{The rating given by %judge $i$ to characteristic $k$ of object $j$} \\ 
%$R_{jk}$ & 
%\parbox[t]{3cm}{$\forall j \in \left\lbrace 1,...,M\right\rbrace$,\newline
%$k \in \left\lbrace 1,...,K\right\rbrace$ } & \parbox[t]{5cm}{The intrinsic value of %characteristic $k$ of object $j$}\\ 
%$\Delta_{ijk}$ & 
%\parbox[t]{3cm}{$\forall i \in \left\lbrace 1,...,N\right\rbrace$,\newline
%$j \in \left\lbrace 1,...,M\right\rbrace$,\newline
%$k \in \left\lbrace 1,...,K\right\rbrace$ }
%& \parbox[t]{5cm}{Bias of rating for the rating given by judge $i$ to characteristic %$k$ of object $j$} \\
%$\mu_{ijk}$ & 
%\parbox[t]{3cm}{$\forall i \in \left\lbrace 1,...,N\right\rbrace$,\newline
%$j \in \left\lbrace 1,...,M\right\rbrace$,\newline
%$k \in \left\lbrace 1,...,K\right\rbrace$ } 
%& \parbox[t]{5cm}{Mean of rating for the rating given by judge $i$ to characteristic %$k$ of object $j$($\mu_{ijk} = Q_{jk} + \Delta_{ijk}$)}\\
%$\sigma_{i_1k_1i_2k_2}^2$ &
%\parbox[t]{3cm}{$\forall i_1,i_2 \in \left\lbrace 1,...,M\right\rbrace$,\newline
%$k_1,k_2 \in \left\lbrace 1,...,K\right\rbrace$ }
% & \parbox[t]{5cm}{Variance of rating for the judge $i$ for characteristic $k$}\\
%$w_{ijk}$ & & \parbox[t]{5cm}{Weight of the rating $ijk$}\\
%\hline 

%\end{tabular} 
%\caption{Notations}\label{table:notation}
%\end{table}
\end{document}